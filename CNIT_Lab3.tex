% Created 2024-06-25 Tue 21:43
% Intended LaTeX compiler: pdflatex
\documentclass[letterpaper]{article}
\usepackage[utf8]{inputenc}
\usepackage[T1]{fontenc}
\usepackage{graphicx}
\usepackage{longtable}
\usepackage{wrapfig}
\usepackage{rotating}
\usepackage[normalem]{ulem}
\usepackage{amsmath}
\usepackage{amssymb}
\usepackage{capt-of}
\usepackage{hyperref}
\usepackage[margin=1in]{geometry}
\usepackage{float}
\author{Group 10}
\date{Wed Jun 19 07:35:38 2024}
\title{CNIT 242 LAB 3 Documentation}
\hypersetup{
 pdfauthor={Group 10},
 pdftitle={CNIT 242 LAB 3 Documentation},
 pdfkeywords={},
 pdfsubject={Documentation and notes for the third and final lab for CNIT 242.},
 pdfcreator={Emacs 29.4 (Org mode 9.6.24)}, 
 pdflang={English}}
\begin{document}

\maketitle
\tableofcontents

\newpage
\section{{\bfseries\sffamily IN-PROGRESS} Lab 3}
\label{sec:orgd1deb7c}
\noindent\textbf{DEADLINE:} \textit{<2024-06-28 Fri>}\\[0pt]
\subsection{Intro video notes}
\label{sec:org18e28d5}
\url{https://purdue.brightspace.com/d2l/le/content/1013650/viewContent/16679580/View}
\subsubsection{Network Setup (0-10:30min)}
\label{sec:orgab108f0}
\begin{itemize}
\item Add new nic to outer VCenter
\item Power cycle the ESXi servers
\begin{itemize}
\item Turn off all the VMs inside the ESXi servers before restarting them
\end{itemize}
\end{itemize}
\begin{enumerate}
\item Network topology
\label{sec:orgd4130e0}
\begin{figure}[H]
\centering
\includegraphics[width=12.5cm]{/home/sam/Screenshots/screenshot_2024-06-19_11-37-58.png}
\caption{Network topology as seen in the introductory video. Purple is the new part. Note that our IPs are different.}
\end{figure}
\end{enumerate}
\subsubsection{Domain controllers (10:30-14:20min)}
\label{sec:orgc618050}
\begin{itemize}
\item Setup new domain controller(SCDC2) using the windows 2019 template and put in on the port group that you just made(the host portion of the netmask will be different)
\item Configure the DNS of (SCDC2) to point to (SCDC1) before you upgrade it to a domain controller
\end{itemize}
\begin{figure}[H]
\centering
\includegraphics[width=12.5cm]{/home/sam/Screenshots/screenshot_2024-06-19_11-53-25.png}
\caption{Screenshot of video part talking about setting up domain controllers.}
\end{figure}
\subsubsection{DNS Configuration (14:20-18:05min)}
\label{sec:orgd193102}
\begin{itemize}
\item DNS settings will need to be changed to address the new site
\end{itemize}
\begin{figure}[H]
\centering
\includegraphics[width=12.5cm]{/home/sam/Screenshots/screenshot_2024-06-19_12-06-48.png}
\caption{DNS settings between sites. Our IPs will be different.}
\end{figure}
\subsubsection{Sites and Subnets (18:05-24:40min)}
\label{sec:org058431f}
\begin{enumerate}
\item Using ``active directory sites and services'' for this
\end{enumerate}
\begin{figure}[H]
\centering
\includegraphics[width=12.5cm]{/home/sam/Screenshots/screenshot_2024-06-19_12-16-57.png}
\caption{Overview of the sites. Each one will have their respective VMs/Servers in them not shown here.}
\end{figure}
\subsubsection{DFS (21:40-29:36min)}
\label{sec:org7075e38}
\begin{itemize}
\item Creating a share using the domain name
\item EX: \texttt{\textbackslash{}\textbackslash{}<domainname>\textbackslash{}Share}
\item needs a DFS sync task
\end{itemize}
\begin{figure}[H]
\centering
\includegraphics[width=12.5cm]{/home/sam/Screenshots/screenshot_2024-06-19_12-29-19.png}
\caption{I don't really understand whats going on here. Will need to do more research. Something about optimizing sharing resources.}
\end{figure}
\subsubsection{IIS (29:37-32:35min)}
\label{sec:orgb0e5172}
\begin{itemize}
\item public webpage
\item private webpage
\begin{itemize}
\item configure so that it authenticates against the users in the domain
\end{itemize}
\item site specific webpage
\begin{itemize}
\item setup so that only the people in that site can access
\end{itemize}
\end{itemize}
\begin{figure}[H]
\centering
\includegraphics[width=12.5cm]{/home/sam/Screenshots/screenshot_2024-06-19_12-36-04.png}
\caption{Setting up webpages that interact with our system in various ways.}
\end{figure}
\subsubsection{WSUS (32:36-END)}
\label{sec:org9883910}
``Windows Software Update Service''
\begin{itemize}
\item Need to make another windows server for this
\begin{itemize}
\item Install WSUS onto it
\end{itemize}
\item We just need the updates for windows 10 and server 2019 and maybe 2022(for SCDC1)
\item You have to tell each of your devices about the WSUS server(so it can get its updates from there)
\begin{itemize}
\item Use group policy for this
\end{itemize}
\end{itemize}
\begin{figure}[H]
\centering
\includegraphics[width=12.5cm]{/home/sam/Screenshots/screenshot_2024-06-19_12-48-41.png}
\caption{Overview the the WSUS configuration.}
\end{figure}
\subsection{Objectives}
\label{sec:org0cfee70}
\subsubsection{Phase I}
\label{sec:org2cd4f05}
\begin{enumerate}
\item {\bfseries\sffamily DONE} Add a new VM port group to your ESXi servers representing a second enterprise location
\label{sec:org123dedf}
\noindent\textbf{DEADLINE:} \textit{<2024-06-19 Wed>}\\[0pt]
\begin{itemize}
\item[{$\boxtimes$}] Put a new physical NIC of your ESXi VM (outer virtualization layer in studentvc) into the port group specified in the IP Configuration Sheet
\item[{$\boxtimes$}] Name your new port group (in your inner virtualization layer on your ESXi servers) descriptively
\item[{$\boxtimes$}] Address all VMs in this new location using the “Secondary IP Addresses” range from the IP Configuration Sheet
Start LAB DOCUMENTATION:
\textbf{Log into the student virtual cluster}
\begin{itemize}
\item Opened Firefox.
\item Entered \emph{studentvc.cit.lcl} in the search bar.
\item Logged in with my student credentials.
\end{itemize}
\textbf{Added new NIC}
\begin{itemize}
\item Found \textbf{ESXi1.2} in the sidebar
\item | Right click | Edit settings | Virtual Hardware | Add New Device | Network Adapter |
\item Selected the new network adapter.
\item | Browse | CNIT24210B | OK
\item Repeated these steps for \textbf{ESXi2.1}
\end{itemize}
\textbf{Power cycle the ESXi servers}
\begin{itemize}
\item Launched web console for \textbf{CNIT24200.Group10.juergens.win10} from the sidebar
\item Opened Firefox.
\item Entered the ip address of \textbf{ESXi1.2}: \emph{44.100.10.191}
\item Entered login information
\item Entered the ip address of \textbf{ESXi2.1}: \emph{44.100.10.192}
\item Entered login information
\item In each ESXi servers respective tab went to | Virtual Machines | Checked checkbox next to Virtual machine | Power off
\item For each ESXi server clicked | Host | Reset | OK
\item In each ESXi servers respective tab went to | Virtual Machines | Checked checkbox next to Virtual machine | Power off
\end{itemize}
\textbf{Configured ESXi servers network settings}
\begin{itemize}
\item Navigated back to \textbf{ESXi1.2}'s web interface
\item Clicked | Networking | Add standard virtual switch |
\item Named it \emph{vSwitch2}
\item Clicked | ADD |
\item Clicked | Port groups | Add port group |
\item Named it \emph{site2 Management}
\item Clicked | ADD |
\item Clicked | Port groups | Add port group |
\item Named it \emph{site2 VM network}
\item Selected vSwitch2 from the Virtual switch drop down menu
\item Clicked | ADD |
\item Selected | Add VMkernal NIC | site2 | IPv4 only |
\item Entered \emph{44.100.60.0} for the Address and \emph{255.255.255.0} for the subnet mask
\item Clicked Create
\item Navigated to \textbf{ESXi2.1}'s web interface
\item Clicked | Networking | Add standard virtual switch |
\item Named it \emph{vSwitch2}
\item Clicked | ADD |
\item Clicked | Port groups | Add port group |
\item Named it \emph{site2 Management}
\item Clicked | ADD |
\item Clicked | Port groups | Add port group |
\item Named it \emph{site2 VM network}
\item Selected vSwitch2 from the Virtual switch drop down menu
\item Clicked | ADD |
\item Selected | Add VMkernal NIC | site2 | IPv4 only |
\item Entered \emph{44.100.60.0} for the Address and \emph{255.255.255.0} for the subnet mask
\item Clicked Create
\end{itemize}
\end{itemize}
\item {\bfseries\sffamily DONE} Deploy a new instance of your Windows Server VM template and implement it as a second domain controller at the new location
\label{sec:org80d0301}
\noindent\textbf{DEADLINE:} \textit{<2024-06-20 Thu>}\\[0pt]
\begin{itemize}
\item[{$\boxtimes$}] Define the new subnet/site in Active Directory
\item[{$\boxtimes$}] Change/move FSMO roles per best practices from lecture
START Lab Procedure Documentation:
\end{itemize}
\textbf{Deploy New instance of your Windows server template}
\begin{itemize}
\item Went to inner virt. enviornment
\end{itemize}
\textbf{Created the Server 2019 VM from template}
\begin{itemize}
\item Inventory | Detestores | SANdatastoreG10 | VMs | VM templates |
\item right clicked \textbf{winServer 2019}
\item Selected new VM from this template
\item Entered \emph{SCDCO2} for virtual machine name
\item | NEXT | 44.100.10.192 | Thin VM Storage Policy | datastore | NEXT | NEXT | FINISH
\end{itemize}
\textbf{Navigated to ESXi2.1's IP address}
\begin{itemize}
\item Found the new server ``SCDCO2''
\item Navigated through VM Hardware| Edit | Network Adapter 1 |Browse | site2 VM network | ok | ok |
\end{itemize}
\textbf{Launched SCDCO2's web console}
\begin{itemize}
\item Logged in as Administrator
\end{itemize}
\textbf{Configured IPv4 settings}
\begin{itemize}
\item Change IPv4 proporties:
\begin{itemize}
\item IP: \emph{44.100.60.10}
\item Subnet Mask: \emph{255.255.255.0}
\item Default Gateway: \emph{44.100.60.1}
\item DNS: \emph{44.100.10.10}
\end{itemize}
\item Verified connectivity by pinging 44.100.10.10
\end{itemize}
\textbf{Installed Active Directory Domain Services}
\begin{itemize}
\item Opened server manager
\item Navigated to | manage | Add roles and features | next | Role-based | Next |
\item Checked Active Directory Domain Services
\item | Next | Next | Next | Install |
\end{itemize}
\textbf{Promoted the server to a domain controller}
\begin{itemize}
\item Clicked the flag icon
\item Clicked | Promote this server to a domain controller |
\item Selected Add a domain controller to an existing domain
\item Entered:
\begin{itemize}
\item Domain: \emph{group10.c242.cit.lcl}
\item Credentials:
\begin{itemize}
\item username: \emph{group10.242.cit.lc1\Administrator}
\item password: \emph{Cnit242!}
\end{itemize}
\end{itemize}
\item Clicked | Next |
\item Entered DRSM password: \emph{Cnit 242!}
\item Clicked | next | next | next | install |
\item Restarted \textbf{SCDCO2}
\end{itemize}
\textbf{Defining a new site and Subnet}
\begin{itemize}
\item | Tools | Active directory Users and computers |
\item Right-clicked Sites
\item Selected new site
\item Named it site 2
\item Selected |Default|Psite link | OK|
\item Right-click Subnets
\item Selected New Subnet
\item Entered this for prefix: \emph{44.100.60.0/24}
\item Navigated to | Default-First-Site-Name | Servers |
\item Right-clicked SCDCO2 | Move | site 2 | ok |
\end{itemize}

\item {\bfseries\sffamily DONE} Install a Windows 10 VM at the new location
\label{sec:org2673ae9}
\noindent\textbf{DEADLINE:} \textit{<2024-06-20 Thu>}\\[0pt]
START Lab Procedure Documentation:

\textbf{Install a Windows 10 VM at the new location}
\begin{itemize}
\item went to the inner vCenter environment
\item Right-clicked \textbf{44.100.10.192}
\item New VM | Create a new virtual machine | Next |
\item Named it \textbf{site2win10}
\item | Next | Next | Thin storage policy | Next | Next |
\item Added 16 GB of Memory
\item Changed disk size to 35 GB, thin provision
\item Clicked | Next | finish |
\item Opened the web console
\item Accepted the defaults for the windows installer wizard
\end{itemize}
\textbf{Configured IPv4 settings for Windows 10 VM}
\begin{itemize}
\item Change IPv4 proporties:
\begin{itemize}
\item IP: \emph{44.100.60.120}
\item Subnet Mask: \emph{255.255.255.0}
\item Default Gateway: \emph{44.100.60.1}
\item DNS: \emph{44.100.60.10}
\item Secondary DNS: \emph{44.100.10.10}
\end{itemize}
\end{itemize}
\textbf{Install updates}
\begin{itemize}
\item Installed VMware tools
\item Went to settings
\item Selected | update | install | restart |
\end{itemize}
\textbf{Join domain}
\begin{itemize}
\item Went to settings
\item Searched \emph{domain} in the search bar
\item Entered \emph{group10} in the domain box
\item Entered user: \emph{Administrator} password: \emph{Cnit242!}
\end{itemize}
\item {\bfseries\sffamily DONE} Implement Microsoft Backup
\label{sec:orgb2aa829}
\noindent\textbf{DEADLINE:} \textit{<2024-06-22 Sat>}\\[0pt]
\begin{itemize}
\item[{$\boxtimes$}] Set up scheduled backups of your domain controllers using Vembu backup
\item[{$\boxtimes$}] \url{https://www.vembu.com/free-windows-servers-backup/}
\item[{$\boxtimes$}] Create a share on a Windows 10 client in your outer virtualization layer to use as a backup target
\begin{itemize}
\item[{$\boxtimes$}] Add a new disk to the Windows 10 client VM to house backups
\end{itemize}
\item[{$\boxtimes$}] To conserve disk space in the lab environment, backup only user files and the domain controller database (DCDB)
\item[{$\boxtimes$}] Follow best practices as defined in lecture for backups
\item[{$\boxtimes$}] You need only keep one week’s worth of data
START LAB Procedure documentation:
\textbf{Install Vembu Backup}
\begin{itemize}
\item Navigated to the outer vCenter virtualization environment
\item Launched \textbf{CNIT24200.Group10.elsner.win10}
\item Downloaded \textbf{Vembu} from www.vembu.com
\item Opened the .exe file that was just downloaded
\item Accepted the defaults in the installer wizar.
\end{itemize}
\textbf{Install Windows Server Backup}
\begin{itemize}
\item Opened \textbf{SCDC01} and \textbf{SCDC02} and performed the following steps on both
\item Opened \textbf{Server Manager}
\item Navigated to | Manage | Add Roles an Features | Next | Next | Next |
\item Checked \textbf{Windows Server Backup} and \textbf{Windows Server Migration Tools}
\item Completed the installer
\end{itemize}
\textbf{Configure Backup}
\begin{itemize}
\item Opened BDR Suite web interface
\item Navigated to | Backup | Configure Backup | Windows | Files \& Folders |
\item Added Block storage repository
\item Entered \emph{$\backslash$\MIDDLESCHOOLPC\TheBackup} for the drive path
\item Entered username: \emph{Administrator} and password: \emph{Cnit242!}
\item Selected | Create | Next | Confirm
\item Checked the local host to backup
\item Added the user files and domain controller database file locations
\item Configured the schedule to backup once every sunday morning at 2am
\end{itemize}
\end{itemize}

\item {\bfseries\sffamily IN-PROGRESS} Printing
\label{sec:org3f2da4d}
\noindent\textbf{DEADLINE:} \textit{<2024-06-23 Sun>}\\[0pt]
\begin{itemize}
\item[{$\square$}] Create a new user and make them a print operator/administrator
\begin{itemize}
\item[{$\square$}] Experiment with controlling the printer and other user’s print jobs
\end{itemize}
\item[{$\square$}] Create a second queue for the printer that prints on legal size paper
\begin{itemize}
\item[{$\square$}] Make the legal size printer available to only members of the administrators group
\end{itemize}
\end{itemize}
\item {\bfseries\sffamily IN-PROGRESS} Clone the Windows Server VM template and implement an IIS web server
\label{sec:orge8332c5}
\noindent\textbf{DEADLINE:} \textit{<2024-06-24 Mon>}\\[0pt]
\url{https://purdue.brightspace.com/d2l/le/content/1013650/viewContent/16679581/View}
\begin{itemize}
\item[{$\square$}] Configure a public web page on the server that anyone can access
\item[{$\square$}] Configure a private web page on the server that only members of the domain can access
\item[{$\square$}] Configure a site-specific web page that only machines on the first VLAN can access
START Lab Procedure Documentation:
\end{itemize}
\textbf{Deploy New instance of your Windows server template}
\begin{itemize}
\item Went to inner virt. enviornment
\end{itemize}
\textbf{Created the Server 2019 VM from template}
\begin{itemize}
\item Inventory | Detestores | SANdatastoreG10 | VMs | VM templates |
\item right clicked \textbf{winServer 2019}
\item Selected new VM from this template
\item Entered \emph{IISserver} for virtual machine name
\item | NEXT | 44.100.10.191 | Thin VM Storage Policy | datastore | NEXT | NEXT | FINISH
\end{itemize}
\textbf{Navigated to ESXi1.2's IP address}
\begin{itemize}
\item Found the newly created server in the sidebar
\item Navigated through VM Hardware| Edit | Network Adapter 1 | Browse | site2 VM network | ok | ok |
\end{itemize}
\textbf{Launched IISserver's web console}
\begin{itemize}
\item Logged in as Administrator
\end{itemize}
\textbf{Configured IPv4 settings}
\begin{itemize}
\item Change IPv4 proporties:
\begin{itemize}
\item IP: \emph{44.100.60.200}
\item Subnet Mask: \emph{255.255.255.0}
\item Default Gateway: \emph{44.100.60.1}
\item DNS: \emph{44.100.10.10}
\item Secondary DNS: \emph{44.100.60.10}
\end{itemize}
\end{itemize}
\textbf{Changed server name and add to domain}
\begin{itemize}
\item Launced server manager
\item | Computer Name | Change |
\item Entered \emph{IISserver} for computer name and \emph{group10.c242.cit.lcl} for domain
\item Restarted to apply changes
\end{itemize}
\textbf{Install IIS}
\begin{itemize}
\item Navigated to | Server Manager | Manage | Add roles and features | Role-based | Next | Next |
\item Checked ``Web Server (IIS)''
\item Finished the installer by accepting the defaults for the remaining installer wizard
\item Navigated to | Tools | Internet Information Services (IIS) Manager
\item 
\end{itemize}
\item {\bfseries\sffamily TODO} Implement DFS across your two domain controllers
\label{sec:org22ddb25}
\noindent\textbf{DEADLINE:} \textit{<2024-06-25 Tue>}\\[0pt]
\begin{itemize}
\item[{$\square$}] Implement a DFS domain namespace
\begin{itemize}
\item[{$\square$}] Change the references to the shares hosting home directories and desktops in
\end{itemize}
\end{itemize}
profiles, folder redirection, and group policy to reference the DFS namespace
share instead of a specific server share
\begin{itemize}
\item[{$\square$}] Implement DFS replication to replicate home directory and desktop data between the
\end{itemize}
domain controllers
\begin{itemize}
\item[{$\square$}] Using DFS and the Active Directory sites tool configure the domain such that the
\end{itemize}
clients access the shares from the domain controller at their site.
\end{enumerate}
\subsubsection{Phase 2}
\label{sec:orgd4b5abe}
\begin{enumerate}
\item {\bfseries\sffamily TODO} Clone the Windows Server VM template and implement a Windows Server Update Services (WSUS) instance
\label{sec:org6d3920f}
\noindent\textbf{DEADLINE:} \textit{<2024-06-26 Wed>}\\[0pt]
\begin{itemize}
\item[{$\square$}] Add a second 80 GB virtual hard drive (thin provisioned) to house downloaded
\end{itemize}
updates
\begin{itemize}
\item[{$\square$}] Configure your clients to pull updates from your WSUS server via a GPO
\end{itemize}
\item {\bfseries\sffamily TODO} Use PowerShell Remote to list/stop/start services on a remote machine
\label{sec:orgc427ce2}
\noindent\textbf{DEADLINE:} \textit{<2024-06-27 Thu>}\\[0pt]
\item {\bfseries\sffamily TODO} Implement VMware Thinapp
\label{sec:orgfea2f70}
\noindent\textbf{DEADLINE:} \textit{<2024-06-28 Fri>}\\[0pt]
\begin{itemize}
\item[{$\square$}] Package Microsoft Visio 7 and deploy to your physical machines using thinapp
\end{itemize}
\end{enumerate}
\subsubsection{Notes}
\label{sec:orgc676ae3}
\begin{enumerate}
\item Add Additional Virtual Machines
\label{sec:orga19d483}
\begin{itemize}
\item New virtual machines may be added to either ESXi server and stored on any available datastore.
\end{itemize}
\item System Backup
\label{sec:org55c1762}
\begin{itemize}
\item Configure the built-in backup solution (MS Backup) to automatically back up the Domain
\end{itemize}
Controller database and all user files on the domain controllers to a share on the physical
Windows 10 client. Implement a media (backup file) rotation scheme per lecture. To schedule the
backups to run automatically you will have to use the utility built into Windows 2012 Server.
\item Printing
\label{sec:org015b93a}
\begin{itemize}
\item Create two print queues for the lab printer – one that duplexes and one that doesn't. Experiment
\end{itemize}
with controlling the printer (start/stop the queue, etc.) and other users’ print jobs. The printer is a
limited, shared resource so minimizing actual printing during this process will help all groups.
Pause the printer to enable multiple jobs to be queued, then re-order/delete print jobs.
\item Distributed File System (DFS)
\label{sec:orgf03db75}
\begin{itemize}
\item Implement a DFS domain namespace such that \texttt{\textbackslash{}\textbackslash{}yourdomain.tld\textbackslash{}dfs} shows available network
\end{itemize}
shares. Reconfigure folder redirection/profiles/etc to refer to the DFS namespace
\texttt{(\textbackslash{}\textbackslash{}yourdomain.tld\textbackslash{}dfs\textbackslash{}share)} rather than the server \texttt{(\textbackslash{}\textbackslash{}server\textbackslash{}share).}
\begin{itemize}
\item Implement DFS replication so that the shares in the server namespace are replicated between the
\end{itemize}
two domain controllers and are available on each. The client at each site should connect to the
server at their site to access the share. The physical server being accessed can be shown by using
the netstat command to show open connections.
Additional information on DFS can be found on Microsoft Technet at \url{https://docs.microsoft.com/en-us/windows-server/storage/dfs-namespaces/dfs-overview}.
\item Windows Server Update Services
\label{sec:orgfc93f5c}
\begin{itemize}
\item Enable Microsoft Windows Server Update Services (WSUS) on a new Windows Server instance.
\end{itemize}
Create, partition, and format a second virtual hard drive to house the update store.
\begin{itemize}
\item Once WSUS is installed, updates will have to downloaded and authorized for your Windows 10
\end{itemize}
and Windows Server hosts. Only download updates for these two operating systems. To save
time, your WSUS server can be configured to get updates from the CIT/NET WSUS server at
wsus.cit.lcl.
\begin{itemize}
\item Clients that are members of the domain can easily be configured to receive updates from the
\end{itemize}
WSUS server automatically using group policy.
\item VMware Thinapp
\label{sec:org62a2612}
Using a clean install of Windows 10, take a snapshot and install Thinapp. Run the Thinapp setup-
capture and install Visio. Concluded setup-capture and save the package to a network location.
Revert to the snapshot and run the package on the computer
\item IP Tables
\label{sec:org6aa26a7}
\begin{itemize}
\item Primary (Site1)
\begin{itemize}
\item 44.100.10.0/24
\end{itemize}
\item Secondary (Site2)
\begin{itemize}
\item 44.100.60.0/24
\end{itemize}
\end{itemize}
\end{enumerate}

\section{{\bfseries\sffamily TODO} Lab 3 Report}
\label{sec:orgbe49f8b}
\noindent\textbf{DEADLINE:} \textit{<2024-07-06 Sat>}\\[0pt]
\end{document}
